\documentclass{article}[12pt]

\usepackage[top=1in,bottom=1in,right=1in,left=1in]{geometry}

\begin{document}


\title{Homework Assignment 4 \\
   \large Stacked Graphs -- Geometry \& Aesthetics
Lee Byron \& Martin Wattenbergr \\}
\date{November 2, 2018}

\author{
   Prathyusha Butti \\
   pbutti@email.arizona.edu
}

\maketitle

\noindent
{\bf 1. What problem is this paper trying to solve?}

{
    This paper is trying to tell us how difficult it is to find a better way to display such large amounts of data retaining the emotional importance of the data.
}




\vspace{2ex}\noindent
{\bf 2. Why is this problem considered a visualization problem?}

{
   We need to represent data as a form of emotion and not only as a way of statistical information. As it connects to movies, people generally feel emotional in this genre and we need to depict that carefully while drawing a visualization. We need a better time series visualization to represent data for over 21 years of 7500 movies.
}




\vspace{2ex}\noindent
{\bf 3. Why is the problem important? }

{
    The problem is important because it tells us the importance of the role of aesthetics in visualization design, and the process and trade-offs necessary to create engaging information graphics. How such large volumes of data require many designs and algorithms to analyze and present an understandable visualization.  
}




\vspace{2ex}\noindent
{ \bf 4. How does this paper contribute to solving the problem? }

{
    They provide a mathematical analysis of how this layered graph relates to traditional stacked graphs and to techniques such as Theme-River, showing how each method is optimizing a different “energy function”. Finally, they discuss techniques for coloring and ordering the layers of such graphs and emphasize the interplay between considerations of aesthetics and legibility. 
    
}


\vspace{2ex}\noindent
{\bf 5. What approaches are used to construct the contributions?}

{
    The authors pay special attention to the response on the web and the role of aesthetics in the appeal of visualizations and perform a detailed analysis of the algorithms that define these graphs.They even suggest that this type of complex layered graph is effective for displaying large data sets to a mass audience. 

}



\vspace{2ex}\noindent
{\bf 6. How are the contributions of the paper evaluated or justified? }

{
    First, as with any information graphic, legibility of the data is critical. They have justified this in section 4.1 by mentioning the problems they have encountered and how it has influenced in design decisions.

    Second, as the reactions to the Listening History and New York Times visualizations show, aesthetics seem to play an important role in the popularity of this type of graphic. They have justified this in section 4.2

    They have analyzed Listening History - Last.FM and NY Times Box Office Revenue graphs in a detailed way considering the above two points. 
    
}



\vspace{2ex}\noindent
{\bf 7. What do you think are this paper's strengths? }

{
    The paper strength's is the process they have discovered the issues (A-F) in section 4.1 and 4.2 and how they have proposed a unified approach to a stacked graph geometry.
    
    They have come up with major 4 ingredients that determine a generalized stacked graph. First, The shape of the overall silhouette is the first ingredient; The second important parameter is the ordering of the layers. Third, labels are important. Finally, color choice is critical, enabling viewers to distinguish different layers and potentially conveying additional data dimensions. They have described algorithms that address each of these four ingredients with respect to the design issues of legibility and aesthetics.

}

\vspace{2ex}\noindent
{\bf 8. What do you think could be improved about this paper? }

{
    Since they have only analyzed stacked graphs, Are there other layouts possible for different data-sets with different layouts?
    How they have concluded on the colors they have used for this visualization? 
    Did they conduct any evaluation process for the conclusion?
    What kind of design rationale have they used to conclude on legibility and aesthetics as main parameters in design process?
    Comparative analysis of this visualization and why it is best in this scenario.
    
}

\vspace{2ex}\noindent
{\bf 9. What future directions do the authors suggest?}

{
    A rigorous content analysis of the comments they have received for the visualization.
    Exploring and studying when and how to compromise between aesthetics and legibility.
    Reordering layers in more depth.
    All these systems have subtly different variations on the techniques for defining the geometry, layout, color,
    and interaction of the graph.How can this be dealt with in more systematic approach?
}

\vspace{2ex}\noindent
{\bf 10. What other future directions would you suggest? }

{
   How can they best show hierarchical information?
   Can they show the same visualization using time histograms?
   They can compare with other visualizations in the same context and let us know why stacked graphs is best to visualize in this context?
   Can they have visualization in focus+context approach where they can zoom more into the visualization without disturbing the layering?
}

\vspace{2ex}\noindent
{\bf 11. What questions do you have about this paper? For example: Were these things you find difficult to understand? Are there details left unanswered? Do you have philosophical questions regarding some of the points made?}

{
    They didn't mention their color choice parameters. Have they conducted any experiments to check with the colors? As many people are color-blinded. Have they taken this into the consideration of selection of colors? How did they conclude on the choice of colors? 
    
    How can the labeling be improved? If we want to take a print of the visualization we can't understand what it represents as there is no interactivity on paper print. So can there be a way to solve this so that when we see a visualization on paper we are able to understand what it represents?
    
    Theme-River Layout was a bit difficult to understand initially for me.

}

\vspace{2ex}\noindent
{\bf 12. How might the concepts or approaches in this paper relate to your course project? }

{
    This paper clearly tells me that aesthetics and legibility are two important factors which must be considered in designing a new visualization. In our project we are planning to create a map visualization which tells us the busiest station with respect to time. This paper tell us how best we can work on the layering and zooming of the map and gives us ideas about the labeling also.

}


\end{document}
