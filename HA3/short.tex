\documentclass{article}[12pt]

\usepackage[top=1in,bottom=1in,right=1in,left=1in]{geometry}

\begin{document}


\title{Homework Assignment 3 \\
	}
\date{October 12, 2018}

\author{
   Prathyusha Butti \\
   pbutti@email.arizona.edu
}

\maketitle

\noindent
{\bf 1. Compare and contrast the Stasko model and van Wijk model regarding the value of visualization. How are they similar and how are they different? }

{

	Van Wijk urges that designers first set up requirements, then generate multiple solutions, then match the
	solutions to the requirements and pick the best one.

	Amar and Stasko have proposed a high-level task taxonomy: expose uncertainty, concretize relationships, formulate
	cause and effect, determine domain parameters, multivariate explanation, and confirm hypotheses. Amar, Eagan, and Stasko have also proposed a categorization of low-level tasks as retrieve value, filter, compute derived value, find extremum, sort, determine range, characterize distribution, find anomalies, cluster, correlate.

	Toward a deeper understanding the role  of interaction Yi, Kang and Stasko have proposed to select, explore, reconfigure, encode, abstract, filter and connect.


    \begin{itemize}
	\item Visualization has to be effective and efficient
	\item Visualization Value = Time + Insights + Ensure + Confidence
	\item T - Ability to minimize the total time needed to
	answer a wide variety of questions about the data
	\item I - Ability to spur and discover insights or
	insightful questions about the data

	\item E - Ability to convey an overall essence
	or take-away sense of the data

	\item C - Ability to generate confidence and trust
	about the data, its domain and context

	\end{itemize}
}




\vspace{2ex}\noindent
{\bf 2. In a design study, what are the typical contributions to the visualization community?}

{
	A design study is a project in which visualization researchers analyze a specific real-world problem faced by domain experts, design a visualization system that supports solving this problem, validate the design, and reflect about lessons learned in order to refine visualization design guidelines.

	By focusing on the process for a real world problem, we can contribute back to the visualization community:
	\begin{itemize}
    \item Problem characterization
    \item Problem abstractions
    \item Visual design
    \item Refinement of visualization process
    \end{itemize}
}




\vspace{2ex}\noindent
{\bf 3. Explain three ways in which a design study can go wrong. }

{ 
    \begin{itemize}
    \item   If you don't have a visualization problem, data, and people
	\item   If you commit to a design too early
    \item	If you don't keep visualization and methodology refinement goals in mind
    \end{itemize}
}




\vspace{2ex}\noindent
{ \bf 4. In a quantitative visualization experiment, how do I increase external validity? }

{
	External validity can be increased:
	\begin{itemize}
	\item   If the results can be generalized
	\item   If the results can be applied to other participants
	\item   If the results can be applied to other tasks
	\item   If the results can be applied to other data
	\item   If there is sufficient variability in the design of the experiment to generalize
	\item  If you increase situations and participants external validity increases.
    \end{itemize}
}

\vspace{2ex}\noindent
{ \bf 5. Name three patterns which cause humans to group items.  }

{
    \begin{itemize}
    \item	Proximity - Things that are close together are perceptually grouped together
	\item   Similarity - Similar elements tend to be grouped together
	\item   Continuity - Visual elements that are smoothly connected or continuous tend to be grouped
	\item   Symmetry - Two symmetrically arranged visual elements are more likely to be perceived as a whole
	\item   Closure - A closed contour tends to be seen as an object
	\item   Relative Size - Smaller components of a pattern tend to be perceived as objects whereas large ones as a background

 	 \end{itemize}

}

\vspace{2ex}\noindent
{ \bf 6. Name a benefit and a weakness of the CIELab color space. }

{   \begin{itemize}
	\item   Benefit : CIElab is perceptually uniform. Distance in color space corresponds to distance in perceptual space
	\item Weakness : CIElab does not take into account contrast effects 
    \end{itemize}
}

\vspace{2ex}\noindent
{ \bf 7. How can a Voronoi tessellation aid interactive selection? }

{
    \begin{itemize}
	\item Voronoi Tessellation defines regions by closest point.
	\item  Benefit of the voronoi version is that it’s easier to get a sense of a region, especially a rather empty region.
    \end{itemize}
}

\end{document}
